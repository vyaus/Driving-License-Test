\documentclass[b5paper]{ctexbook}
\usepackage{xcolor}    % 提供颜色支持
\usepackage{colortbl}  % 提供表格背景色支持
\usepackage{tikz}
\usepackage{geometry}
\usepackage{fancyhdr}
\usepackage{multirow}
\usepackage{titlesec}
\usepackage{xcolor}
\usepackage{ulem}
\usepackage[colorlinks=true, linkcolor=red, urlcolor=gray]{hyperref}
\usepackage{cleveref}
\titleformat*{\section}{\bfseries\Large\raggedright}

\newcommand{\docname}{驾考笔记}
\newcommand{\companyname}{\href{https://github.com/vyaus}{vyaus}}

\pagestyle{fancy}
%\fancyhf{} % 清除所有默认的页眉和页脚
\fancyfoot{} % 清除默认的页脚
\fancyfoot[RO, LE]{\thepage} % 奇数页左侧,偶数页右侧
\fancyfoot[LO]{\docname} % 文档名称
\fancyfoot[RE]{\companyname} % 文档名称

\geometry{margin=2cm}
\title{\docname}
\author{\companyname}
%\titlepic{\includegraphics[width=0.8\textwidth]{sections/images/cover.png}}
\date{\today}
\ctexset{
    today = big
}
\let\cleardoublepage\clearpage % 清除章节末尾的空白页
\begin{document}
    \maketitle

    \tableofcontents
    \fangsong
    \chapter{科目一}
    \section{扣分项}
    \subsection{超速、超员、超载扣分}

\begin{table}[htbp]
\centering
\caption{超速扣分}
    \begin{tabular}{|c|c|c|c|c|}
        \hline
        \textbf{车型}  & \textbf{普通道路}  & \textbf{扣分}  & \textbf{高速公路}  & \textbf{扣分} \\
        \hline
        校车 & $10\% \sim 20\% $ & 1 & $< 20\%$ & 6 \\
        \cline{2-5}
        中型以上载客载货汽车 & $20\% \sim 50\%$ & 6 & \multirow{2}{*}{$\ge 20\%$} & \multirow{2}{*}{12} \\
        \cline{2-3}
        危险物品运输车 & $\ge 50\%$ & 9 & & \\
        \hline
        \multirow{2}{*}{其它机动车 } & $20\% \sim 50\%$ & 3 & $20\% \sim 50\%$ & 6\\
        \cline{2-5}
         & $\ge 50\%$ & 6 & $\ge 50\%$ & 12 \\
        \hline
    \end{tabular}
    \label{tab:speeding}
\end{table}
\begin{table}[htbp]
\centering
\caption{超员扣分}
    \begin{tabular}{|c|c|c|c|c|}
        \hline
        \textbf{超员比例}  & \textbf{$< 20\%$} & \textbf{$20\% \sim 50\%$} & \textbf{$50\% \sim 100\%$} & \textbf{$\ge 100\%$}\\
        \hline
         校车 & \multirow{3}{*}{6} & \multicolumn{3}{c|}{} \\
        \cline{1-1}
        公路客运汽车 & & \multicolumn{3}{c|}{12}  \\
        \cline{1-1}
        旅游客运汽车 & & \multicolumn{3}{c|}{} \\
        \hline
        载货汽车 & \multicolumn{4}{c|}{3(违规载客)} \\
        \hline
        7座以上 & & 6 & 9 & 12 \\
        \cline{1-1} \cline{3-5}
        其它 & & 3 & 6 & 12 \\
        \hline
    \end{tabular}
    \label{tab:overcrowding}
\end{table}

\begin{table}[htbp]
\centering
\caption{载货汽车超载扣分}
    \begin{tabular}{|c|c|}
        \hline
        \textbf{超载比例}  & \textbf{扣分} \\
        \hline
        $< 30\%$ & 1 \\
        \hline
        $30\% \sim 50\%$ & 3 \\
        \hline
        $\ge 50\%$ & 6 \\
        \hline
    \end{tabular}
    \label{tab:overloading}
\end{table}

\subsection{其它扣分项}

\subsubsection*{扣12分}

\begin{itemize}
    \item 饮酒后驾驶机动车的;
    \item 造成致人轻伤以上或者死亡的交通事故后逃逸,尚不构成犯罪的;
    \item 使用伪造、变造的机动车号牌、行驶证、驾驶证、校车标牌或者使用其他机动车号牌、行驶证的;
    \item 驾驶机动车在高速公路、城市快速路上倒车、逆行、穿越中央分隔带掉头的;
    \item 代替实际机动车驾驶人接受交通违法行为处罚和记分牟取经济利益的。
\end{itemize}

\subsubsection*{扣9分}

\begin{itemize}
    \item 驾驶机动车在高速公路或者城市快速路上违法停车的;
    \item 驾驶未悬挂机动车号牌或者故意遮挡、污损机动车号牌的机动车上道路行驶的;
    \item 驾驶与准驾车型不符的机动车的;
    \item 未取得校车驾驶资格驾驶校车的;
    \item 连续驾驶\dotuline{中型以上载客汽车、危险物品运输车辆}超过4小时未停车休息或者停车休息时间少于20分钟的。
\end{itemize}

\subsubsection*{扣6分}

\begin{itemize}
    \item 驾驶机动车载运爆炸物品、易燃易爆化学物品以及剧毒、放射性等危险物品,未按指定的时间、路线、速度行驶或者未悬挂警示标志并采取必要的安全措施的;
    \item 驾驶机动车运载超限的不可解体的物品,未按指定的时间、路线、速度行驶或者未悬挂警示标志的;
    \item 驾驶机动车运输危险化学品,未经批准进入危险化学品运输车辆限制通行的区域的;
    \item 驾驶机动车不按交通信号灯指示通行的;
    \item 机动车驾驶证被暂扣或者扣留期间驾驶机动车的;
    \item 造成致人轻微伤或者财产损失的交通事故后逃逸,尚不构成犯罪的;
    \item 驾驶机动车在高速公路或者城市快速路上违法占用应急车道行驶的。
\end{itemize}

\subsubsection*{扣3分}

\begin{itemize}
    \item 驾驶机动车在高速公路或者城市快速路上不按规定车道行驶的;
    \item 驾驶机动车不按规定\textcolor{red}{超车、让行},或者在高速公路、城市快速路\dotuline{以外的}道路上\textcolor{red}{逆行}的;
    \item 驾驶机动车遇前方机动车停车排队或者缓慢行驶时,借道超车或者占用对面车道、\textcolor{red}{穿插}等候车辆的;
    \item 驾驶机动车有拨打、接听手持电话等妨碍安全驾驶的行为的;
    \item 驾驶机动车行经人行横道不按规定减速、停车、避让行人的;
    \item 驾驶机动车不按规定避让校车的;
    \item 驾驶不按规定安装机动车号牌的机动车上道路行驶的;
    \item 在道路上车辆发生故障、事故停车后,不按规定使用灯光或者设置警告标志的;
    \item 驾驶未按规定定期进行安全技术检验的公路客运汽车、旅游客运汽车、危险物品运输车辆上道路行驶的;
    \item 驾驶校车上道路行驶前,未对校车车况是否符合安全技术要求进行检查,或者驾驶存在安全隐患的校车上道路行驶的;
    \item 连续驾驶\dotuline{载货汽车}超过4小时未停车休息或者停车休息时间少于20分钟的;
    \item 驾驶机动车在高速公路上行驶低于规定最低时速的。
\end{itemize}

\subsubsection*{扣1分}

\begin{itemize}
    \item 驾驶机动车不按规定\textcolor{red}{会车},或者在高速公路、城市快速路\dotuline{以外的}道路上不按规定\textcolor{red}{倒车、掉头}的;
    \item 驾驶机动车不按规定使用灯光的;
    \item 驾驶机动车违反禁令标志、禁止标线指示的;
    \item 驾驶机动车载货长度、宽度、高度超过规定的;
    \item 驾驶未按规定定期进行安全技术检验的公路客运汽车、旅游客运汽车、危险物品运输车辆\dotuline{以外的}机动车上道路行驶的;
    \item 驾驶擅自改变已登记的结构、构造或者特征的载货汽车上道路行驶的;
    \item 驾驶机动车在道路上行驶时,机动车驾驶人未按规定系安全带的;
    \item 驾驶摩托车,不戴安全头盔的。
\end{itemize}

    \section{《河南省道路安全条例》罚款项}
    \subsection{罚100}

\noindent 驾驶机动车有下列情形之一的,处一百元罚款:

\begin{itemize}
    \item 违反分道行驶规定的;
    \item 未按照交通标志、标线指示或者交通警察指挥行驶的;
    \item 违反倒车规定的;
    \item 违反牵引挂车规定的;
    \item 违反依次交替通行规定的;
    \item 违反试车规定的;
    \item 违反灯光使用规定的;
    \item 违反故障机动车牵引规定的;
    \item 向道路上抛撒物品的。
\end{itemize}

\noindent 机动车载人、载物有下列情形之一的,处一百元罚款:

\begin{itemize}
    \item 非公路客运车辆载人超过核定人数未达到百分之二十的;
    \item 驾驶摩托车违反规定载人的;
    \item 驾驶拖拉机违反规定载人的;
    \item 运载超限的不可解体物品,未按照规定行驶的。
\end{itemize}

\subsection{罚150}

\noindent 驾驶机动车有下列情形之一的,处一百五十元罚款:

\begin{itemize}
    \item 驾驶安全设施不齐全的车辆上道路行驶的;
    \item 驾驶机件不符合机动车国家安全技术标准的机动车上道路行驶的;
    \item 服用国家管制的精神药品或者麻醉药品后驾驶机动车的;
    \item 患有妨碍安全驾驶机动车的疾病驾驶机动车的;
    \item 使用他人机动车驾驶证的。
\end{itemize}

\subsection{罚200}

\noindent 驾驶机动车有下列情形之一的,处二百元罚款:

\begin{itemize}
    \item 未悬挂机动车号牌或者未取得机动车临时通行牌证、未按照临时通行牌证载明的有效期限行驶的;
    \item 未按照规定安装号牌的;
    \item 故意遮挡或者污损机动车号牌的;
    \item 改变车身颜色、更换发动机、更换车身或者车架,未在规定的时间内办理变更登记的;
    \item 货运机动车及其挂车的车身或者车厢后未喷涂放大的牌号或者放大的牌号不清晰的;
    \item 大、中型客运机动车未按照规定喷涂核定人数或者经营单位名称的;
    \item 驾驶未按照规定期限进行安全技术检验的机动车的;
    \item 安装、使用影响道路交通安全技术监控设施正常使用的装置或者材料的。
\end{itemize}

\noindent 驾驶机动车有下列情形之一的,处二百元罚款:

\begin{itemize}
    \item \dotuline{逆向}行驶的;
    \item 违反规定在专用车道内行驶的;
    \item 违反交通信号灯指示的;
    \item 违反规定超车的;
    \item 违反规定\dotuline{变更车道}的;
    \item 违反规定\dotuline{会车}的;
    \item 违反规定掉头的;
    \item 违反限制或者禁止通行规定的;
    \item 行经人行横道遇行人通过时,未停车让行的;
    \item 非公路客运车辆载人超过核定人数达到百分之二十以上的;
    \item 货运机动车违反规定附载作业人员的;
    \item 运载爆炸物品、易燃易爆化学品以及剧毒、放射性等危险物品未按照规定行驶的;
    \item 通过铁路道口,违反交通信号或者管理人员指挥的。
\end{itemize}

\noindent 驾驶机动车有下列情形之一的,处二百元罚款:

\begin{itemize}
    \item 拨打接听手持电话、观看电视的;
    \item 下陡坡时故意熄火或者空档滑行的;
    \item 连续驾驶营运车辆超过四个小时,未停车休息或者停车休息时间少于二十分钟的;
    \item 在高速公路和同方向划有二条以上机动车道的道路上行驶,长时间占用超车道的;
    \item 警车、消防车、救护车、工程救险车违反规定使用警报器、标志灯具的;
    \item 违反规定停放车辆,影响其他车辆、行人通行的;
    \item 城市公共汽车违反规定停靠的
\end{itemize}

\noindent 遇前方道路受阻或者前方车辆排队等候、缓慢行驶时,驾驶机动车有下列情形之一的,处二百元罚款:

\begin{itemize}
    \item 违反规定进入路口的;
    \item 违反规定在人行横道或者网状线区域内停车等候的;
    \item 借道超车的;
    \item 占用对面车道的;
    \item 穿插等候车辆的;
    \item 进入非机动车道、人行道行驶的。
\end{itemize}

\noindent 驾驶机动车发生故障或者事故,有下列情形之一的,处二百元罚款:

\begin{itemize}
    \item 未按照规定开启危险报警闪光灯的;
    \item 未按照规定设置警告标志的;
    \item 夜间未开启示廓灯和后位灯的;
    \item 发生交通事故后,未按照规定撤离现场,造成交通堵塞的;
    \item 机动车发生故障后尚能移动,未移至不妨碍交通地点的。
\end{itemize}

\noindent 上道路学习驾驶或者实习期间驾驶机动车,有下列情形之一的,处二百元罚款:

\begin{itemize}
    \item 未按照指定路线、时间学习驾驶或者教练车乘坐无关人员的;
    \item 在实习期间内驾驶禁止驾驶的机动车的。
\end{itemize}

机动车驾驶人在高速公路、城市快速路或者其他封闭的机动车专用道违反道路交通安全法律、法规关于道路通行规定的,处二百元罚款。法律、法规和本条例另有规定的除外。

\subsection{罚其它金额}

机动车在道路上行驶超过规定时速百分之五十的,处二百元以上二千元以下罚款;机动车在高速公路上行驶,超过规定时速百分之五十的,处一千元以上二千元以下罚款。

在高速公路上停车上下乘客、装卸货物的,对驾驶员处五百元以上二千元以下罚款。

在高速公路上发生事故,不及时报警致使交通堵塞的,对机动车驾驶人处五百元以上一千元以下罚款。

% \vspace{0.5em} % 调整分割线上方间距
% \noindent\textcolor{gray}{\hrulefill} % 自适应长度的灰色实线
% \vspace{0.5em} % 调整分割线下方间距
% 
% 公路客运车辆载客超过核定人数或者违反规定载货的,对机动车驾驶人按下列规定处罚:
% 
% \begin{itemize}
    % \item 超过核定人数未达百分之二十的,处三百元罚款;
    % \item 超过核定人数百分之二十以上未达百分之五十的,处五百元以上一千元以下罚款;
    % \item 超过核定人数百分之五十以上的,处一千五百元罚款;
    % \item 违反规定载货的,处五百元以上一千元以下罚款。
% \end{itemize}
% 
% 货运机动车超过核定载质量或者违反规定载客的,对机动车驾驶人按下列规定处罚:
% 
% \begin{itemize}
    % \item 超过核定载质量未达百分之三十的,处三百元罚款;
    % \item 超过核定载质量百分之三十以上未达百分之五十的,处五百元以上一千元以下罚款;
    % \item 超过核定载质量百分之五十以上的,处一千五百元罚款;
    % \item 货运机动车违反规定载客的,处五百元以上一千元以下罚款。
% \end{itemize}


\begin{table}[htbp]
\centering
\caption{客车超员、货车超载及违规罚款}
    \begin{tabular}{|c|c|c|}
        \hline
        \textbf{罚款}  & \textbf{公路客运车超员} & \textbf{货运机动车超载} \\
        \hline
        300 & $< 20\%$ & $< 30\%$ \\
        \hline
        $500 \sim 1000$ & $20\% \sim 50\%$ & $30\% \sim 50\%$ \\
        \hline
        1500 & $\ge 50\%$ & $\ge 50\%$ \\
        \hline
        $500 \sim 1000$ & 违规载货 & 违规载客 \\
        \hline
    \end{tabular}
    \label{tab:penalty}
\end{table}

\noindent 有下列情形之一的,按以下规定处罚:

\begin{itemize}
    \item 未取得机动车驾驶证、机动车驾驶证被吊销或者被暂扣期间驾驶\uline{营运汽车}的,处一千元以上一千五百元以下罚款;
    \item 未取得机动车驾驶证、机动车驾驶证被吊销或者被暂扣期间驾驶\dotuline{非营运汽车}的,处五百元以上一千元以下罚款;
    \item 将\uline{营运汽车}交由未取得机动车驾驶证、机动车驾驶证被吊销或者被暂扣期间的人驾驶的,处一千元以上一千五百元以下罚款;
    \item 将\dotuline{非营运汽车}交给未取得机动车驾驶证、机动车驾驶证被吊销或者被暂扣期间的人驾驶的,处五百元以上一千元以下罚款。
\end{itemize}

\vspace{0.5em} % 调整分割线上方间距
\noindent\textcolor{gray}{\hrulefill} % 自适应长度的灰色实线
\vspace{0.5em} % 调整分割线下方间距

擅自停用公共停车场(库)或者改变公共停车场(库)用途的,由公安机关交通管理部门责令限期恢复,逾期不恢复的,从停用或者改变用途之日起按每日每平方米三元处以罚款。

擅自设置或者占用、撤销道路临时停车泊位,或者在机动车停车泊位内设置停车障碍的,由公安机关交通管理部门处五百元罚款。

    \section{准驾车型及代号与年龄限制}
    \noindent

\begin{table}[htbp]
    \begin{tabular}{|l|c|c|l|}
        \hline
        \textbf{准驾车型}  & \textbf{代号}  & \textbf{年龄限制}  & \textbf{准予驾驶的其他车型代号} \\
        \hline
        \textcolor{blue}{大型客车} & A1 & \multirow{2}{*}{$22 \sim 63$} & A3、B1、B2、C1、C2、C3、C4、M \\
        \cline{1-2} \cline{4-4}
        \textcolor{blue}{重型牵引挂车} & A2 & & B1、B2、C1、C2、C3、C4、C6、M \\
        \hline
        城市公交车 & A3 & \multirow{3}{*}{$20 \sim 63$} & C1、C2、C3、C4 \\
        \cline{1-2} \cline{4-4}
        \textcolor{blue}{中型客车} & B1 & & \multirow{2}{*}{C1、C2、C3、C4、M} \\
        \cline{1-2}
        大型货车 & B2 & & \\
        \hline
        小型汽车 & C1 & \multirow{2}{*}{$\ge 18$} & C2、C3、C4 \\
        \cline{1-2} \cline{4-4}
        小型自动挡汽车 & C2 & & \\
        \hline
        低速载货汽车 & C3 & \multirow{2}{*}{$18 \sim 63$} & C4 \\
        \cline{1-2} \cline{4-4}
        三轮汽车 & C4 & & \\
        \hline
        残疾人专用小型自动挡载客汽车 & C5 & $\ge 18$ & \\
        \hline
        \textcolor{blue}{轻型牵引挂车} & C6 & $20 \sim 70$ & \\
        \hline
        普通三轮摩托车 & D & \multirow{2}{*}{$18 \sim 70$} & E、F \\
        \cline{1-2} \cline{4-4}
        普通二轮摩托车 & E & & F \\
        \hline
        轻便摩托车 & F & $\ge 18$ & \\
        \hline
        轮式专用机械车 & M & $18 \sim 63$ & \\
        \hline
        无轨电车 & N & \multirow{2}{*}{$20 \sim 63$} & \\
        \cline{1-2} \cline{4-4}
        有轨电车 & P & & \\
        \hline
    \end{tabular}
    \label{tab:Model_code}
\end{table}

已持有机动车驾驶证,申请增加准驾车型的,应当在本记分周期和申请前最近一个记分周期内没有记满12分记录。申请增加轻型牵引挂车、中型客车、重型牵引挂车、大型客车准驾车型的,还应当符合下列规定:

\begin{enumerate}
    \item 申请增加轻型牵引挂车准驾车型的,已取得驾驶小型汽车、小型自动挡汽车准驾车型资格一年以上;
    \item 申请增加中型客车准驾车型的,已取得驾驶城市公交车、大型货车、小型汽车、小型自动挡汽车、低速载货汽车或者三轮汽车准驾车型资格二年以上,并在申请前最近连续二个记分周期内没有记满12分记录;
    \item 申请增加重型牵引挂车准驾车型的,已取得驾驶中型客车或者大型货车准驾车型资格二年以上,或者取得驾驶大型客车准驾车型资格一年以上,并在申请前最近连续二个记分周期内没有记满12分记录;
    \item 申请增加大型客车准驾车型的,已取得驾驶城市公交车、中型客车准驾车型资格二年以上、已取得驾驶大型货车准驾车型资格三年以上,或者取得驾驶重型牵引挂车准驾车型资格一年以上,并在申请前最近连续三个记分周期内没有记满12分记录。
\end{enumerate}

正在接受全日制驾驶职业教育的学生,已在校取得驾驶小型汽车准驾车型资格,并在本记分周期和申请前最近一个记分周期内没有记满12分记录的,可以申请增加大型客车、重型牵引挂车准驾车型。

    \section{容易混淆的扣分、罚款项}
    \begin{table}[htbp]
    \begin{tabular}{|l|c|c|}
        \hline
        \textbf{违反……规定}  & \textbf{扣分}  & \textbf{罚款} \\ 
        \hline
        使用灯光 & \multirow{3}{*}{1} & \multirow{3}{*}{100} \\
        \cline{1-1}
        禁令标志、禁止标线 &  &  \\
        \cline{1-1}
        倒车(普通道路) &  &  \\
        \hline
        掉头(普通道路) & \multirow{2}{*}{1} & \multirow{2}{*}{200} \\
        \cline{1-1}
        会车 & & \\
        \hline
        超车 & \multirow{3}{*}{3} & \multirow{3}{*}{200} \\
        \cline{1-1}
        让行 & & \\
        \cline{1-1}
        逆行 & & \\
        \hline
        交通信号灯 & 6 & 200 \\
        \hline
    \end{tabular}
    \label{tab:Confusing_Items}
\end{table}

    \chapter{科目二}

\noindent
后视镜角度调整:

\begin{enumerate}
    \item 后门把手调到镜子上边缘
    \item 车身占镜子三分之一(左侧后视镜,侧头可以看到左后轮)
\end{enumerate}

    \section{倒车入库}

    \noindent
右倒库点位(回一圈):

\begin{enumerate}
    \item 挂入倒挡,看右侧车窗右下角蓝白杆漏出,向右打满方向
    \item 观察右侧后视镜,车库口虚线的最后一条实线剩下1/3时,方向盘回一圈
    \item 继续观察右后视镜,当库脚消失时,立刻再次向右打满方向
    \item 观察左后视镜,底角出现后,方向盘回正
    \item 观察左后视镜,其下沿与地面白线一指时停车
\end{enumerate}

\noindent
左倒库点位(回一圈):

\begin{enumerate}
    \item 挂入倒挡,看左侧后视镜下沿与地上白线有一指距离,向左打满方向
    \item 观察左侧后视镜,车库口最后一条虚线消失时,方向盘回一圈
    \item 继续观察左后视镜左后轮,当其走到左库线延长线时,再次向左打满方向
    \item 观察右后视镜,库底角出现后,方向盘回正
    \item 观察左后视镜,其下沿与地面白线一指时停车
\end{enumerate}

    \section{侧方停车}
    \begin{enumerate}
    \item 方向盘最高点对准地上的中心线驶入项目
    \item 身体与右侧蓝白杆对齐时停车挂倒挡
    \item 右后视镜中车库的第一节虚线中的空白部分消失时,向右打满方向
    \item 在左后视镜中库底角出现时回正方向
    \item 在左后视镜中,看到车轮即将压到虚线时,向左打满方向(或者在左后视镜前门把手进线一半时)
    \item 观察右后视镜,车身与库边线基本平行时停车
    \item 打左转向灯、挂一档向外走
    \item 左侧引擎盖与道路左边线对齐时,回正方向
    \item 雨刮器突起与道路左边线对齐时,向右打一圈
    \item 在车身基本回正时,回正方向盘
\end{enumerate}

    \section{曲线行驶}
    \begin{enumerate}
    \item 让车身与项目边线对齐后驶入项目
    \item 在左侧大灯与右边线对齐时(也可观察左侧后视镜边缘与左边线对齐这个点位),向左打一圈方向,不断微调方向使两者始终对齐
    \item 在两个弯道的衔接处,左侧大灯会快速的向左侧边线靠齐,两者靠齐后回正方向
    \item 待雨刮器凸起与左边线对齐时,向右大一圈方向
    \item 侧头看窗外左侧的白线,调节方向盘使其始终与下窗沿前1/3处对齐
\end{enumerate}

    \section{坡道定点停车与起步}
    \begin{enumerate}
    \item 方向盘最高点对着两根虚线中右数第二根实线的左侧向前开
    \item 即将上坡时,稍微松一点离合
    \item 在左后视镜内下沿刚超出实线时,轻踩离合、踩刹车、拉手刹
    \item 大约3秒后,放下手刹,松离合至车身抖动,然后慢松刹车。如果溜车则立即踩下刹车,踩离合的脚的内测稍微翘起,再次慢松刹车
    \item 在车子开始前进后,向左打一点方向,完全放松刹车
    \item 下坡时,离合踩到底,轻踩刹车
    \item 接近坡底时,松离合至半联动,慢松刹车
\end{enumerate}

    \section{直角转弯}
    \begin{enumerate}
    \item 打左转向灯,让方向盘最高点沿着地上箭头右侧向前行驶
    \item 在前门开关与拐弯处直角的横线对齐时,向左打满方向
    \item 在左后视镜可以看到直角时,关掉转向灯
    \item 随着车身回正,慢慢回正方向盘
\end{enumerate}
\end{document}